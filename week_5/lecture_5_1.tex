\documentclass{beamer}
\usepackage[russian]{babel}
\usetheme{metropolis}

\usepackage{amsthm}
\setbeamertemplate{theorems}[numbered]

\setbeamercolor{block title}{use=structure,fg=white,bg=gray!75!black}
\setbeamercolor{block body}{use=structure,fg=black,bg=gray!20!white}

\usepackage[T2A]{fontenc}
\usepackage[utf8]{inputenc}

\usepackage{hyphenat}
\usepackage{amsmath}
\usepackage{graphicx}

\AtBeginEnvironment{proof}{\renewcommand{\qedsymbol}{}}{}{}

\title{
Микроэкономика-I
}
\author{
Павел Андреянов, PhD
}

\begin{document}

\maketitle

\section{План}

\begin{frame}{План}

Первая часть лекции посвящена функциям издержек (не путать с функцией расходов) и минимизация издержек. В частности, мы выведем выпуклость функции издержек по уровню производства, также некоторые приложения.

Вторая часть лекции посвящена более аксиоматической постановке вопроса. Появятся аксиомы производителя подобно тому, что мы видели в теории потребителя. Эта часть лекции нужна, также, для обоснования вогнутости производственных функций, на которые мы опирались в первой части лекции.

\end{frame}


\section{Функция издержек}

\begin{frame}{Функция издержек}

Точно так же, как в теории потребителя поведение агента задавалось функцией полезности, в теории производителя поведение агента задается функцией издержек $TC(p, y)$.

Здесь $y$ – это количество произведенного товара, а $\vec p$ – это вектор цен на факторы производства. Не путайте функцию издержек с функцией расходов $E(\vec p, \bar U)$. Обратите внимание, что я буду теперь везде использовать векторные обозначения.

\end{frame}

\begin{frame}{Функция издержек}

У нас будут два набора цен: $\vec q$ на конечные товары и $\vec p$ на факторы. Также будут два набора координат: $\vec y$ для конечных товаров и $\vec x$ для факторов. Но в первой части лекции я для простоты буду говорить только об одном произведенном товаре.

Таким образом, суммарная прибыль фирмы производителя можно записать в виде:
$$ \pi(\vec p, q, y) = - FC - \vec p \cdot \vec x + q y = q y - TC(p, y)$$

Постарайтесь не запутаться.

\end{frame}

\begin{frame}{Функция издержек}

\begin{definition}
\textbf{Функция издержек} (общие издержки) TC определяется как суммарные издержки на все факторы производства, связанные с эффективным производством $y$ единиц конечного товара, плюс, возможно, фиксированные издержки (которые обозначаются FC).
\end{definition}

Что значит \textbf{эффективность}? Это значит, что из всех производственных планов выбирается тот, который минимизирует издержки.
\end{frame}

\begin{frame}{Функция издержек}

Tо есть функция издержек – это, на самом деле, целевая функция следующей оптимизационной задачи:
$$ TC(p, y) = \min_{\vec x} ( FC + \vec p \cdot \vec x ) \quad s.t. \quad F(\vec x) \geqslant y,$$
где $\vec x$ – это вектор использованных факторов производства, а $F$ – это \textbf{производственная функция}, которая переводит факторы производства $\vec x$ в конечный товар $y$.

То есть функция издержек – это тоже огибающая.

\end{frame}

\begin{frame}{Функция издержек}

Нас интересует вопрос выпуклости (или вогнутости) функции издержек по ценам факторов $\vec q$ и, отдельно, по уровню производства $y$. 

Для простоты опустим фиксированные издержки.

\begin{lemma}
\textbf{Функция издержек} $TC(p,y)$
\begin{itemize}
\item вогнута по ценам факторов $\vec p$
\item выпукла по уровню производства $y$,
\end{itemize}

если сама производственная функция $F$:
$$
F(\vec x) = y
$$
вогнута (именно вогнута, квази недостаточно).
\end{lemma}

\end{frame}

\begin{frame}{Функция издержек}

Найдем седловую точку лагранжиана. Но надо аккуратно записать его так, чтобы при выходе за ограничение вас наказали бесконечно большим положительным значением $\lambda$
$$ \min_{\vec x \geqslant 0} \max_{\lambda \geqslant 0} \mathcal{L}, \quad \mathcal{L} = \vec p \cdot \vec x - \lambda \cdot (F(\vec x) - y)$$

Заметим, что лагранжиан линеен по параметрам задачи $\vec p$ и $y$. Это значит чуть больше, чем обычно...

\end{frame}

\begin{frame}{Функция издержек}

Это значит, что опорные функции - линейны по параметрам.

Огибающие линейного семейства опорных функций всегда либо выпуклы, либо вогнуты, в зависимости от того, с какой стороны происходит огибание. Также надо помнить, что огибание происходит именно в пространстве параметров, потому что по $\vec x$ опорные функции вовсе не линейные. 

Также надо понять, с какой стороны происходит огибание.

\end{frame}

\begin{frame}{Функция издержек}

$$ \min_{\vec x \geqslant 0} \max_{\lambda \geqslant 0} \mathcal{L}, \quad \mathcal{L} = \vec p \cdot \vec x - \lambda \cdot (F(\vec x) - y)$$

Поскольку мы минимизируем $\vec p \cdot \vec x$, по ценам факторов огибание получается снизу, поэтому функция издержек вогнута по $\vec p$.

\end{frame}

\begin{frame}{Функция издержек}

$$ \min_{\vec x \geqslant 0} \max_{\lambda \geqslant 0} \mathcal{L}, \quad \mathcal{L} = \vec p \cdot \vec x - \lambda \cdot F(\vec x) + \lambda \cdot y$$

С другой стороны, по уровням производства огибание происходит сверху, главное не перепутать знак, поэтому функция издержек выпукла по $y$.

\end{frame}

\begin{frame}{Функция издержек}

Осталось убедиться, что вообще можно пользоваться методом Лагранжа здесь. Ответ простой - задача выпуклая, потому что $F$ вогнутая. 

Но почему недостаточно квазивогнутости $F$? 

Ответ - линейные члены в лагранжиане не могут поломать вогнутость, но они могут сломать квазивогнутость.

\end{frame}

\section{Максимизация прибыли}

\begin{frame}{Максимизация прибыли}

Теперь, когда у фирмы есть на руках выпуклая функция издержек, она может промаксимизировать свою прибыль по уровню производства.

$$ \max_{y} \pi, \quad \pi(\vec p, q, y) = q y - TC(\vec p, y)$$

Обратите внимание, что эта задача - выпуклая.

\end{frame}

\begin{frame}{Максимизация прибыли}

Условия первого порядка гласят, что оптимальный уровень производства во внутренней точке $y^{\ast}$ удовлетворяет:
$$ q = MC(\vec p, y^{\ast}).$$

\begin{definition}
\textbf{Маржинальные издержки} $MC$ определяются как приращение функции издежек при увеличении производства, то есть: 
$$ MC = \frac{\partial}{\partial y} TC(\vec p, y).$$
\end{definition}

\end{frame}

\begin{frame}{Максимизация прибыли}

Получается, что фирма максимизирует прибыль в два шага: сначала для каждого потенциального уровня производства она оптимально подбирает ресурсы, а затем оптимизирует по уровню производства. 

Конечно же, прибыль можно было максимизировать сразу:
$$ \max_{x,y} (- \vec p \cdot \vec x + q y) \quad s.t. \quad F(\vec x)  \geqslant y,$$
но тогда не было бы так интрересно.

\end{frame}

\section{Труд, капитал и Кобб-Дуглас}

\begin{frame}{Труд, капитал и Кобб-Дуглас}

Типичная постановка задачи фирмы – это когда есть два фактора производства: $k$ - капитал и $l$ - труд. Можно сказать, что рыночные цены этих факторов – это $r$ - цена аренды капитала и $w$ - зарплата соответственно. 

Тогда задача фирмы:
$$ \max_y q y - TC(y), \quad TC(y) = \min_{k,l} rk + wl, \quad s.t. \quad F(k,l) \geqslant y$$

\end{frame}

\begin{frame}{Труд, капитал и Кобб-Дуглас}

Введем дополнительные обозначения

\begin{definition}
\textbf{Предельная отдача на капитал} MPK и \textbf{предельная отдача на труд} MPL – это частные производные производственной функции по капиталу и труду соответственно:
$$MPK = \frac{\partial}{\partial k}F(k,l), \quad MPL = \frac{\partial}{\partial l}F(k,l)$$
\end{definition}

\end{frame}

\begin{frame}{Труд, капитал и Кобб-Дуглас}

Легко видеть из метода Лагранжа, что в оптимуме, верно:
$$ r = \lambda \cdot MPK, \quad w = \lambda \cdot MPL \quad \Rightarrow \quad \frac{r}{w} = \frac{MPK}{MPL},$$
ведь это всего лишь условия первого порядка для минимизации издержек.

Было бы удобно, если наша производственная функция обладала свойством, позволяющим относительно легко считать MPK и MPL. 

Оказывается, такая функция есть, и она называется Кобб-Дуглас.

\end{frame}

\begin{frame}{Труд, капитал и Кобб-Дуглас}

\begin{definition}
Производственная функция называется \textbf{Кобб-Дуглас}, если 
$$ F(k,l) = k^{\alpha} l^{\beta}.$$
\end{definition}

Легко видеть, что у Кобба-Дугласа:
$$ MPK = \alpha \frac{y}{k}, \quad  MPL = \beta \frac{y}{l}$$

\end{frame}

\begin{frame}{Труд, капитал и Кобб-Дуглас}

Другими словами,
$$ r k^{\ast} = \alpha \cdot \lambda y, \quad w l^{\ast} = \beta \cdot \lambda y$$

то есть общие расходы фирмы распределяются между капиталом и трудом в пропорциях $\alpha, \beta$. Таким образом, можно, например, откалибровать производственную функцию, обладая доступом к нехитрым налоговым отчетностям фирм.

\end{frame}

\begin{frame}{Труд, капитал и Кобб-Дуглас}

Далее, легко видеть, что 
$$(r k)^{\alpha} = (\alpha \cdot \lambda y)^{\alpha}, \quad (w l)^{\beta} = (\beta \cdot \lambda y)^{\beta}$$
подставляя в производственную функцию, мы получаем
$$ r^{\alpha}w^{\beta} y = \alpha^{\alpha} \beta^{\beta} \lambda y^{\alpha + \beta},$$
откуда легко вычисляется множитель Лагранжа $\lambda^{\ast}$.

\end{frame}

\begin{frame}{Труд, капитал и Кобб-Дуглас}

Наконец, функция расходов вычисляется как:
$$ TC(r,w,y) = r k^{\ast} + w l^{\ast} = (\alpha + \beta) \lambda^{\ast} y$$
от которой мы, конечно, ожидаем, что она будет выпуклой.

Вопрос: при каких значениях $\alpha, \beta$, функция издержек выпуклая? 

Вопрос: Что произойдет, если $\alpha + \beta = 1$?

\end{frame}

\section{Разные спросы}

\begin{frame}{Разные спросы}

Рассмотрим задачу минимизации издержек
$$ \min_{\vec x} \vec p \cdot \vec x \quad s.t. \quad F(\vec x) \geqslant y,$$

\begin{definition}
Назовем условным спросом $\tilde x(p, y)$ на факторы производства - решение задачи минимизации издержек, а обычным спросом
$$x^{\ast}(p, q) = \tilde x(p, y^{\ast}(p, q)).$$
полное решение задачи максимизации прибыли.
\end{definition}

\end{frame}

\section{Краткосрочная перспектива}

\begin{frame}{Краткосрочная перспектива}

В краткосрочной перспективе, некоторые факторы производства нельзя оптимизировать. В зависимости от страны, это может быть либо труд, либо капитал. В США рынок труда очень динамичен, поэтому считается, что капитал зафиксирован. Во многих странах Европы уволить сотрудника, наоборот, гораздо сложнее, чем избавиться от капитала. 

И обычный, и условный спрос, можно найти в краткосрочном периоде. Это означает, что тот фактор, что менять нельзя, необходимо зафиксировать на каком-то уровне. Например, мы можем зафиксировать капитал $k$ на уровне $\hat k$.

\end{frame}

\begin{frame}{Краткосрочная перспектива}

Далее, необходимо перерешать все так, будто $\hat k$ – это параметр задачи.

\begin{definition}
Назовем условным спросом в краткосрочном периоде $\tilde x_{k}(p, y, \hat k)$ на факторы производства - решение задачи минимизации издержек, а обычным спросом в краткосрочном периоде
$$x_k^{\ast}(p, q, \hat k) = \tilde x_k(p, y^{\ast}(p, q, \hat k), \hat k).$$
\end{definition}
Как связаны эти четыре спроса?

\end{frame}

\begin{frame}{Краткосрочная перспектива}

Обычный спрос - это условный спрос, в который подставили оптимальный уровень производства. А условный спрос - это условный спрос в краткосрочном периоде, в который подставили оптимальную условную $k$.

Также, обычный спрос - это обычный спрос в краткосрочном периоде, в который подставили оптимальную $k$.

Соответственно, вы можете решать задачу постепенно: сначала найти все в краткосрочном периоде, а потом дополнительно прооптимизировать по фактору, который был зафиксирован.

\end{frame}

\section{Убывающая отдача от масштаба}

\begin{frame}{Убывающая отдача от масштаба}

На самом деле, можно заметить, что как бы ни была устроена минимизация издержек, нам абсолютно необходимо, чтобы в задаче максимизации прибыли:
$$ \max_{y} \pi, \quad \pi(\vec p, q, y) = q y - TC(\vec p, y)$$
функция издержек была (желательно строго) выпуклой, иначе можно получить бесконечную прибыль.

Фирмы, обладающие вогнутой производственной функцией, обладают также выпуклой по $y$ функцией издержек, это мы доказали.

\end{frame}

\begin{frame}{Убывающая отдача от масштаба}

На практике это означает, что когда фирма растет, ее общая эффективность постепенно падает, то есть имеет место \textbf{убывающая отдача от масштаба}.

Считается, что все фирмы сначала проходят период быстрого роста, от стартапов и бутиков до компаний средних размеров, и затем испытывают сложности при дальнейшем расширении. Выходя за пределы своих локальных рынков, они принимают корпоративную структуру и становятся медленными и неповоротливыми, теряя эффективность.

\end{frame}

\begin{frame}{Убывающая отдача от масштаба}

Есть, конечно, исключения. Например, компания Google давно вышла за пределы своего штата и, даже, страны. Это говорит от том, что технология обработки поисковых запросов, скорее всего, обладает возрастающей отдачей от масштаба.

\end{frame}

\section{Оптимальное распределение производства}

\begin{frame}{Оптимальное распределение производства}

Предположим, что у нас есть два завода, обладающие производственными технологиями:
$$ y_1 = F_1(\vec x), \quad y_2 = F_2(\vec x).$$

Как эффективно разделить уровень производства $y = y_1 + y_2$ и чему будет равна функция издержек $TC(y)$? 

Как это соотносится с индивидуальными функциями издержек заводов $TC_1(y_1)$ и $TC_2(y_2)$?

\end{frame}

\begin{frame}{Оптимальное распределение производства}

Чтобы ответить на этот вопрос, выпишем лагранжиан:
$$ \mathcal{L} = TC(y_1) + TC(y_2) - \lambda (y_1 + y_2 - y)$$

То есть мы минимизируем суммарные издержки так, чтобы достичь определенного суммарного уровня производства.

\end{frame}

\begin{frame}{Оптимальное распределение производства}

Выпишем условия первого порядка:
$$ MC(y_1) = \lambda = MC(y_2), \quad y_1 + y_2 = y.$$

Таким образом, мы доказали следующее утверждение:
\begin{lemma}
Эффективное производство устроено так, что маржинальные издержки равны друг другу.
\end{lemma}

\end{frame}

\begin{frame}{Оптимальное распределение производства}

К примеру, если у нас есть выпуклые издержки $TC_1 = y^2$ и $TC_2 = y^3$, то необходиму решить систему:

$$ 2y_1^{\ast} = 3(y_2^{\ast})^2, \quad y = y_1^{\ast} + y_2^{\ast}$$

и затем определить функцию издержек двух заводов, как:

$$ TC(y) := TC_1(y_1^{\ast}) + TC_2(y_2^{\ast}).$$

\end{frame}

\begin{frame}{Оптимальное распределение производства}

Обратите внимание, что мы не перерешиваем для каждого завода, как правильно закупить факторы производства $\vec x$, а только пользуемся их функциями издержек. 

Это сильный ход, потому что мы не потребовали производственную функцию $F_i$ каждого завода, а воспользовались более простым объектом $TC_i$, который проще откалибровать.

Это настоящая экономика.

\end{frame}

\section{Производственные цепочки}

\begin{frame}{Производственные цепочки}

Предположим, что у нас есть производственная функция
$$ y = F(k, l)$$
и цены факторов производства равны $r, w$ соответственно. Мы уже знаем, как решать такую задачу. 

Однако, предположим, что мы обладаем также технологией с убывающей отдачей, которая позволяет нам производить $k,l$ со следующими функциями издержек: $$ TC^K(k), \quad TC^L(l).$$ 

Как решать такую задачу?

\end{frame}

\begin{frame}{Производственные цепочки}

Если мы покупаем $k,l$ на рынке, то мы платим $rk$ и $wl$, условия первого порядка нам известны. Если мы производим $k,l$ сами, то мы платим $TC^K(k) + TC^L(l)$. 

Сравним Лагранжианы:
\begin{gather*} 
\mathcal{L}^1 = TC^K(k) + TC^L(l) - \lambda (F(k,l) - y)\\
\mathcal{L}^2 = rk + wl - \lambda (F(k,l) - y).
\end{gather*}


Если случится так, что старые $k^{\ast}, l^{\ast}$ такие, что $$ TC^K(k^{\ast}) + TC^L(l^{\ast}) < rk^{\ast} + wl^{\ast}$$

то очевидно, что дешевле произвести все самому, чем покупать на рынке. Но сколько именно?

\end{frame}

\begin{frame}{Производственные цепочки}

Интуиция подсказывает, что из-за убывающей отдачи от масштаба, вы захотите произвести какое-то количество $\hat k, \hat l$ сами, а остальное купить на рынке. 

Давайте запишем Лагранжиан:
$$
\mathcal{L} = TC^K(\hat k) + TC^L(\hat l) + r(k-\hat k) + w(l - \hat l) - \lambda (F(k,l) - y).
$$

\end{frame}

\begin{frame}{Производственные цепочки}
Тогда условия первого порядка будут:
$$ MC^K(\hat k) = r, \quad MC^L(\hat l) = w, \quad \frac{r}{w} = \frac{MPK}{MPL}.$$
Мы доказали еще одно любопытное свойство.

\begin{lemma}
Если фактор производства $x_i$ можно либо купить по цене $p_i$ либо (вогнуто) произвести самостоятельно с функцией расходов $TC^i(x_i)$, то он производится согласно условиям первого порядка:
$$ p_i = MC^i(x_i),$$

а все остальное закупается по рыночным ценам.
\end{lemma}
\end{frame}

\section{Экспорт и потребление}

\begin{frame}{Экспорт и потребление}
Предположим, что фирма, производящая товар $\vec y$ с функцией издержек $TC(\vec y)$, может либо отправить его на экспорт (продать свой товар на рынке) по цене $\vec q$, либо потребить его сама (раздать рабочим, установить в офисе) с вогнутой полезностью $U(\vec y)$. 

Как будет выглядеть оптимальное поведение фирмы?
\end{frame}

\begin{frame}{Экспорт и потребление}

Обозначим внутреннее потребление товара за $\hat y$, тогда:
$$ \pi = U(\hat y) + \vec q \cdot (\vec y - \hat y) - TC(\vec y).$$

Условия первого порядка гласят:
$$ \nabla U = \vec q, \quad \vec q = \nabla T(\vec y),$$

таким образом...
\end{frame}

\begin{frame}{Экспорт и потребление}
таким образом...
\begin{lemma}
	Если товар можно потребить внутри фирмы, то он производится в том же объеме, как если бы такой возможности не было. Доля товара, потребленного внутри фирмы, совпадает с решением задачи потребителя, как если бы он просто покупал этот товар в магазине.
\end{lemma}
\end{frame}

\section{Кривые производственных возможностей}

\begin{frame}{Кривые производственных возможностей}

До сих пор мы моделировали производство так, что из нескольких факторов, например, труд и капитал, производится один единственный конечный продукт по технологии 

$$ F(\vec x) \geqslant y,$$

где $F$ - вогнутая производственная функция. 

\end{frame}

\begin{frame}{Кривые производственных возможностей}

Альтернативный подход к моделированию производства - это когда много конечных продуктов $\vec y$ производится из одного абстрактного фактора $x$, в этом случае технология описывается
$$ x \geqslant G(\vec y),$$

где $G$ - функция, линии уровня которой описывают всевозможные комбинации конечных товаров, которые можно произвести с фиксированным количеством фактора $x$. 

Эти линии уровня называются \textbf{кривыми производственных возможностей}.
\end{frame}

\begin{frame}{Кривые производственных возможностей}
Попробуем сформулировать задачу минимизации издержек:
$$ \min_{\vec x \geqslant 0} \max_{\lambda \geqslant 0} \mathcal{L}, \quad \mathcal{L} = p x - \lambda (x - G(\vec y))$$

Легко видеть, что для повторения тех же шагов, что с вогнутой производственной функцией $F$, нам тут необходимо, чтобы $G$ была квазивыпуклой по $y$. 
\end{frame}

\begin{frame}{Кривые производственных возможностей}

Выпуклой, потому что поменялся знак перед интересующей нас функцией в Лагранжиане. Квази, потому что, в отличие от предыдущего Лагранжиана:

$$ \min_{\vec x \geqslant 0} \max_{\lambda \geqslant 0} \mathcal{L}, \quad \mathcal{L} = \vec p \cdot \vec x - \lambda \cdot (F(\vec x) - y)$$

в котором присутствовала линейная деформация по $x$, способная поломать квазивогнутость, в случае КПВ такой деформации нет.
\end{frame}

\begin{frame}{Кривые производственных возможностей}
\begin{lemma}
\textbf{Функция издержек}

\begin{itemize}
\item вогнута по ценам факторов $\vec p$
\item выпукла по уровню производства $y$,
\end{itemize}

если функция $G$, задающая кривые производственных возможностей:
$$x = G(\vec y)$$

квази-выпукла.
\end{lemma}
\end{frame}

\section{Перерыв}

\end{document}